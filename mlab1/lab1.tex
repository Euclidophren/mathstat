\documentclass[a4paper, 12pt]{article}

%%% Работа с русским языком
%\usepackage{cmap}                   % поиск в PDF
%\usepackage{mathtext}               % русские буквы в формулах
\usepackage[T2A]{fontenc}          % кодировка
\usepackage[english, russian]{babel}    % локализация и переносы
\usepackage{color}                  % цветные буковки



\usepackage[top=20mm, bottom=20mm, left=30mm, right=15mm]{geometry}

%\usepackage{hyperref}

\usepackage{amsmath, amsfonts, amssymb, mathtools} 



\usepackage{graphicx}

\graphicspath{{img/}{../../graphics/}}



\usepackage{amsthm}

\theoremstyle{definition}
\newtheorem{defn}{Определение}[section]
\newtheorem*{rem}{Замечание}



%%% Format

% Математическое ожидание\
\newcommand{\Expect}{%
	\mathsf{M}}

\renewcommand{\Variance}{%
	\mathsf{D}}

% Математическое ожидание
\newcommand{\m}[1]{%
	\ensuremath{M\!#1}}

% Математическое ожидание X
\newcommand{\mx}{%
	\ensuremath{M\!X}}

\newcommand{\mxx}{%
	\ensuremath{M\!X^2}}

% Математическое ожидание Y
\newcommand{\my}{%
	\ensuremath{M\!Y}}

\newcommand{\myy}{%
	\ensuremath{M\!Y^2}}

% Дисперсия
\newcommand{\disp}[1]{%
	\ensuremath{D\!#1}}

% Дисперсия X
\newcommand{\dx}{%
	\ensuremath{D\!X}}

% Дисперсия Y
\newcommand{\dy}{%
	\ensuremath{D\!Y}}

% Следовательно
\newcommand{\Rarrow}{%
	\ensuremath{\;\Rightarrow\;}}

% Бесконечная последовательность 
\newcommand{\infseq}[3]{%
	\ensuremath{#1_#2, \dots, #1_#3, \dots}\ }

% Бесконечная последовательность X_1, ... X_n, ...
\newcommand{\infseqX}{%
	\infseq{X}{1}{n}}

\usepackage{listings, listingsutf8}

\lstset{
	language = Matlab,
	frame = single,
	showstringspaces=false,
	%basicstyle=\ttfamily,
	basewidth={0.55em,0.55em},
	literate={а}{{\selectfont\char224}}1
	{б}{{\selectfont\char225}}1
	{в}{{\selectfont\char226}}1
	{г}{{\selectfont\char227}}1
	{д}{{\selectfont\char228}}1
	{е}{{\selectfont\char229}}1
	{ё}{{\"e}}1
	{ж}{{\selectfont\char230}}1
	{з}{{\selectfont\char231}}1
	{и}{{\selectfont\char232}}1
	{й}{{\selectfont\char233}}1
	{к}{{\selectfont\char234}}1
	{л}{{\selectfont\char235}}1
	{м}{{\selectfont\char236}}1
	{н}{{\selectfont\char237}}1
	{о}{{\selectfont\char238}}1
	{п}{{\selectfont\char239}}1
	{р}{{\selectfont\char240}}1
	{с}{{\selectfont\char241}}1
	{т}{{\selectfont\char242}}1
	{у}{{\selectfont\char243}}1
	{ф}{{\selectfont\char244}}1
	{х}{{\selectfont\char245}}1
	{ц}{{\selectfont\char246}}1
	{ч}{{\selectfont\char247}}1
	{ш}{{\selectfont\char248}}1
	{щ}{{\selectfont\char249}}1
	{ъ}{{\selectfont\char250}}1
	{ы}{{\selectfont\char251}}1
	{ь}{{\selectfont\char252}}1
	{э}{{\selectfont\char253}}1
	{ю}{{\selectfont\char254}}1
	{я}{{\selectfont\char255}}1
	{А}{{\selectfont\char192}}1
	{Б}{{\selectfont\char193}}1
	{В}{{\selectfont\char194}}1
	{Г}{{\selectfont\char195}}1
	{Д}{{\selectfont\char196}}1
	{Е}{{\selectfont\char197}}1
	{Ё}{{\"E}}1
	{Ж}{{\selectfont\char198}}1
	{З}{{\selectfont\char199}}1
	{И}{{\selectfont\char200}}1
	{Й}{{\selectfont\char201}}1
	{К}{{\selectfont\char202}}1
	{Л}{{\selectfont\char203}}1
	{М}{{\selectfont\char204}}1
	{Н}{{\selectfont\char205}}1
	{О}{{\selectfont\char206}}1
	{П}{{\selectfont\char207}}1
	{Р}{{\selectfont\char208}}1
	{С}{{\selectfont\char209}}1
	{Т}{{\selectfont\char210}}1
	{У}{{\selectfont\char211}}1
	{Ф}{{\selectfont\char212}}1
	{Х}{{\selectfont\char213}}1
	{Ц}{{\selectfont\char214}}1
	{Ч}{{\selectfont\char215}}1
	{Ш}{{\selectfont\char216}}1
	{Щ}{{\selectfont\char217}}1
	{Ъ}{{\selectfont\char218}}1
	{Ы}{{\selectfont\char219}}1
	{Ь}{{\selectfont\char220}}1
	{Э}{{\selectfont\char221}}1
	{Ю}{{\selectfont\char222}}1
	{Я}{{\selectfont\char223}}1
}

\newcommand{\biglisting}[1]{%
	\lstinputlisting[numbers=left]{#1}%
}

\begin{document}

\thispagestyle{empty}

\begin{center}
    \Large
    Московский государственный технический университет имени~Н.\,Э.\,Баумана
\end{center}

{\large
\noindent
Факультет: Информатика и системы управления\\[2mm]
\noindent
Кафедра: Программное обеспечение ЭВМ и информационные технологии\\[2mm]
\noindent

Дисциплина: Математическая статистика
\vspace{1.5cm}}

\begin{center}
    \Large
    \textbf{Лабораторная работа №1} \\
    \textbf{Гистограмма и эмпирическая функция распределения} \\
\end{center}
\vfill

\hfill\begin{minipage}{0.35\textwidth}
    Выполнила: Покасова А.И.\\
    Группа:ИУ7-61 \\
    Вариант: 16
\end{minipage}
\vfill

\begin{center}
    Москва, 2019 г.
\end{center}

\newpage
\section{Постановка задачи}

\paragraph{Цель работы:} построение гистограммы и эмпирической функции распределения.

\paragraph{Содержание работы}

\begin{enumerate}
	\item Для выборки объёма $n$ из генеральной совокупности $X$ реализовать в виде программы на ЭВМ
	\begin{itemize}
		\item вычисление максимального значения $M_{\max}$ и минимального значения $M_{\min}$;
		\item вычисление размаха $R$ выборки;
		\item вычисление оценок $\hat{\mu}$ и $S^2$ математического ожидания $\Expect X$ и дисперсии $\Variance X$;
		\item группировку значений выборки в $m = [\log_2 n] + 2$ интервала;
		\item построение на одной координатной плоскости гистограммы и графика функции плотности распределения вероятностей нормальной случайной величины с математическим ожиданием $\hat{\mu}$ и дисперсией $S^2$;
		\item построение на другой координатной плоскости графика эмпирической функции распределения и функции распределения нормальной случайной величины с математическим ожиданием $\hat{\mu}$ и дисперсией $S^2$.
	\end{itemize}
	\item Провести вычисления и построить графики для выборки из индивидуального варианта.
\end{enumerate}
\newpage
\section{Отчёт}

\subsection{Формулы для вычисления величин}

\paragraph{Количество интервалов}
\begin{equation}
m = [\log_2 n] + 2
\end{equation}


\paragraph{Минимальное значение выборки}

\begin{equation}
M_{\min} = \min \{ x_1, \dots, x_n\}, \quad \text{где}
\end{equation}
\begin{itemize}
	\item $(x_1, \dots, x_n)$ --- реализация случайной выборки.
\end{itemize}


\paragraph{Максимальное значение выборки}

\begin{equation}
M_{\max} = \max \{ x_1, \dots, x_n\}, \quad \text{где}
\end{equation}
\begin{itemize}
	\item $(x_1, \dots, x_n)$ --- реализация случайной выборки.
\end{itemize}


\paragraph{Размах выборки}

\begin{equation}
R = M_{\max} - M_{\min}, \quad \text{где}
\end{equation}
\begin{itemize}
	\item $M_{\max}$ --- максимальное значение выборки;
	\item $M_{\min}$ --- минимальное значение выборки.
\end{itemize}


\paragraph{Оценка математического ожидания}

\begin{equation}
\hat{\mu}(\vec{X}) = \overline{X} = \frac{1}{n} \sum_{i = 1}^{n} X_i\,.
\end{equation}


\paragraph{Исправленная оценка дисперсии}

\begin{equation}
S^2(\vec{X}) = \frac{n}{n - 1}\hat{\sigma}^2(\vec{X}) = \frac{1}{n - 1}\sum_{i = 1}^{n} (X_i - \overline{X})^2\,.
\end{equation}



\subsection{Эмпирическая плотность и гистограмма}

\begin{defn}
	\emph{Эмпирической плотностью распределения  выборки $\vec{x}$} называют функцию
	\begin{equation}
	f_n(x) =
	\begin{cases}
	\frac{n_i}{n \, \Delta}, &x \in J_i,\; i = \overline{1, m};\\
	0, &\text{иначе}.
	\end{cases}, \quad \text{где}
	\end{equation}
	\begin{itemize}
		\item $J_i,\, i = \overline{1; m}$, --- полуинтервал из $J = [x_{(1)}, x_{(n)}]$, где 
		\begin{align}
		&x_{(1)} = \min\{ x_1, \dots, x_n \}, &x_{(n)} = \max\{ x_1, \dots, x_n \};
		\end{align}
		при этом все полуинтервалы, кроме последнего, не содержат правую границу т.\,е.
		\begin{align}
		&J_i = [ x_{(1)} + (i-1)\Delta, x_{(1)} + i\Delta), \quad i = \overline{1, m-1};
		\\
		&J_m = [ x_{(1)} + (m-1)\Delta, x_{(1)} + m\Delta];
		\end{align}
		\item $m$ --- количество полуинтервалов интервала $J = [x_{(1)}, x_{(n)}]$;
		\item $\Delta$ --- длина полуинтервала $J_i$, $i = \overline{1, m}$ равная
		\begin{equation}
		\Delta = \frac{x_{(n)} - x_{(1)}}{m} = \frac{|J|}{m};
		\end{equation}
		\item $n_i$ --- количество элементов выборки в полуинтервале $J_i$, $i = \overline{1, m}$;
		\item $n$ --- количество элементов в выборке.
		
	\end{itemize}
\end{defn}

\begin{defn}
	График функции $f_n(x)$ называют гистограммой.
\end{defn}

\subsection{Эмпирическая функция распределения}


\begin{defn}
	Эмпирической функцией распределения, отвечающей выборке $\vec{x}$ называют функцию
	\begin{equation}
	F_n\colon \mathbb{R} \to \mathbb{R}, \qquad F_n(x) = \frac{n(x, \vec{x})}{n},
	\end{equation} 
	где $n(x, \vec{x})$ --- количество элементов выборки $\vec{x}$, которые меньше $x$.
\end{defn}


\section{Листинг программы}

\biglisting{lab1.m}


\section{Результаты расчётов}

\begin{align*}
M_{\min} &= 9.54; \\
M_{\max} &= 15.21; \\
R &= 5.67; \\
\hat{\mu}(\vec{X}_n) &= 11.9532; \\
S^2(\vec{X}_n) &= 1.1818.
\end{align*}

\noindent 
Интервальная группировка значений выборки при $m = 8$:
\[
[9.54;10.25), 5 ,	[10.25;10.96), 17,	[10.96;11.67) ,28,	[11.67;12.38), 24,	[12.38;13.08), 28,
\]
\[
[13.08;13.79), 12,	[13.79;14.50), 5,	[14.50;15.21], 1.
\]



\section{Графики}

\subsection{Гистограмма и график функции плотности распределения вероятностей нормальной случайной величины с математическим ожиданием $\hat{\mu}$ и дисперсией $S^2$}


\begin{figure}[h]
	\centering
	\includegraphics[width=0.7\textwidth]{lab1fig1.png}
	\caption{Гистограмма и график функции плотности распределения нормальной случайной величины.}
\end{figure}



\newpage
\subsection{График эмпирической функции распределения и функции распределения нормальной случайной величины с математическим ожиданием $\hat{\mu}$ и дисперсией $S^2$}

\begin{figure}[h]
	\centering
	\includegraphics[width=0.5\textwidth]{lab1fig2.png}
	\caption{График эмпирической функции распределения и функции распределения нормальной случайной величины.}
\end{figure}

\end{document}